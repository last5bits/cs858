Using automatic repair tools to fix security vulnerabilities and other bugs can potentially save companies great amount of resources. However, current Generate-and-validate automatic repair approaches often produce incorrect patches due to imperfections in their search space; prior work suggests to tackle this problem by creating ``targeted'' search spaces---the search spaces aimed at fixing specific types of vulnerabilities or aimed at particular domains. In this work, we conducted a manual examination of \numvuln Android vulnerabilities to discover how many of those vulnerabilities can be potentially fixed by existing tools (GenProg and SPR). Out of total \numvuln vulnerabilities, 59 are in the search space of GenProg and 24 are in SPR's. To improve the search space of SPR, we proposed a new search space that contains 44 correct patches and has a more promising prioritization of transformations.

\section{Limitations \& Threats to validity}

The concept of search space is relevant only to search-based automatic repair tools (e.g., SPR~\cite{long2015staged}, Prophet~\cite{long2015prophet} or GenProg~\cite{le2012systematic}).
The research community works on other approaches as well; for example, \emph{semantic-based} automatic repair (DirectFix~\cite{directfix}, Angelix~\cite{mechtaev2016angelix}), in which a repair produced by employing symbolic execution and constraint solving.

% Some patches may contain code unrelated to the vulnerability
Previous research shows that a portion of developer commits contains changes that are irrelevant to the target bug (e.g., quick refactoring).
Ngueyen et al~\cite{nguyen2013filtering} reported that 11~--~39\% of bug-fixing commits contain irrelevant changes and we believe that for vulnerability-fixing commits this number should be similar.
To mitigate this issue, a more rigorous empirical study should be conducted, in which every vulnerability is precisely examined and relevant changes are manually distinguished from irrelevant ones; we leave this as future work.

% Human bias in the empirical study
Since the empirical study is performed by the authors manually, there is a risk of human bias.
We mitigated this risk by following a specific procedure when deciding whether a particular patch can be repaired by a tool.

% Limited number of vulnerabilities
The results of our study might not generalize to all the vulnerabilities or all the mobile software systems due to the empirical nature of the study.
We mitigate this by examining a large number of vulnerabilities (\numvuln).

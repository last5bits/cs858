% get rid of a widow heading
\newpage

\section{Conclusions}
\label{section:conclusion}

Automatic repair tools are unable to fix bugs correct patches for which are outside of their search space. However, as the prior work shows~\cite{long2016analysis}, blind extensions of search space without looking at the problem domain leads to fewer correct patches produced. In the current work, we conducted an empirical study of \numvuln Android vulnerabilities to (1) assess the search space of GenProg and SPR in terms of numbers of correct patches, and (2) obtain an insight on how to improve existing search spaces to better target \emph{security vulnerabilities} in \emph{mobile software}. Empirical study showed that majority of Android vulnerabilities cannot be fixed by GenProg or SPR. With the insight obtained during the manual examination of developer patches, we outlined a search space that contains fixes for 44 vulnerabilities (compared to the previous 20) and has a more promising prioritization of transformations. Using this search space should improve effectiveness of automatic repair tools that aim at fixing vulnerabilities in Android.

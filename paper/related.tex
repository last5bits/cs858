\section{Related work}

Automatic repair of specifically mobile software is its infancy; the most related work is by Azim~et al.~\cite{azim2014towards}, their approach detects programs' crashes, immediately patches the bytecode of the program (to be certain that it won't crash again) and rolls the program state to the nearest activity. Automatic repair of desktop applications received more attention; the tool Prophet~\cite{long2015prophet} is a state-of-the-art technique that prioritizes potential fixes inside the search space by learning what correct \emph{developer} fixes look like. This direction is promising, however it highly depends on the data to learn from, which is not always present. Our approach of targeted search spaces is orthogonal to the one of Prophet.

BovInspector~\cite{bovinspector} is a tool for automatic repair of buffer overflow vulnerabilities; its workflow consists of several stages: detecting potential buffer overflows with static analysis, constructing a CFG and performing a reachability analysis; the reachability information is subsequently used to guide symbolic execution and mitigate the inherent problem of path explosion. After a buffer overflow is confirmed by symbolic execution, BovInspector employs three strategies to fix an overflow: introduce an if-condition, change an API function (i.e., \texttt{strcpy} to \texttt{strncpy}) or expand the buffer. One of the limitations of BovInspector is its focus on only buffer overflows, while a search-based repair can target a more wide range of vulnerabilities.
